\documentclass[a4paper, 12pt]{report}
\usepackage{graphicx}
\usepackage{pgf}
\usepackage{pgfpages}
\pgfpagesdeclarelayout{boxed}
{
 \edef\pgfpageoptionborder{0pt}
}
{
 \pgfpagesphysicalpageoptions
 {%
 logical pages=1,%
 }
 \pgfpageslogicalpageoptions{1}
 {
 border code=\pgfsetlinewidth{2pt}\pgfstroke,%
 border shrink=\pgfpageoptionborder,%
 resized width=.95\pgfphysicalwidth,%
 resized height=.95\pgfphysicalheight,%
 center=\pgfpoint{.5\pgfphysicalwidth}{.5\pgfphysicalheight}%
 }%
}
\pgfpagesuselayout{boxed}
\title{\textbf{\Huge{PROGRAMMING} LANGUAGES\\finding the most popular one}}
\date{}
\author{\\ \\ \\ \\ \\ \Large{\bftext{VRUND GHANSHYAMBHAI} DOBARIYA} \\\large{Roll no. CE029 }\\
\large{SID no. 22CEUOS146} \\ \\ \\ B.Tech(CE) Sem-1 \\ \large{Software Workshop LAB 05}}
\begin{document}
\maketitle
\begin{center}
\includegraphics[scale=2.0]{ddu logo}
\end{center}
\\
\\
\\
\\
\begin{centering}
\large{\textbf{Dharmsinh Desai University,
Nadiad}} \\
\\
\\
{\textbf{Faculty of Technology}}\\
\\
\\
{\textbf{Department of Computer Engineering \\}}
\end{centering}
\newpage
\begin{center}
\begin{center}
\includegraphics[scale=1.0]{ddu logo}
\end{center}
\textbf{\Huge{\underline{\bftext{CERTIFICATE}}}} \\ 
\end{center} \\
\Large{
This is to certify that \textbf{Software Workshop} project entitled \textbf{"PROGRAMMING LANGUAGES : finding the most popular one"} is the bonafide report of work carried our by \\
\begin{center}
\textbf{{VRUND GHANSHYAMBHAI DOBARIYA}}
\end{center}\\
of department of Computer Engineering, Sem-I, of Academic year 2022-2023,under supervision and 
guidance of \\
\begin{center}
\textit{Prof. sanjay bhimani} 
\end{center}
}
\tableofcontents
\chapter{Introduction}

\section{\large{why do we need different tyres of programming languages?}}\\
 \large{The world of Programming Languages is very dynamic. Every company is launching their own programming language which can 
cater their specific demand and requirement. In this paper, we discussed various popular rankings proposed by different organizations to decide 
most popular language on basis of various factors like number of Google Trends searches, number of job advertisements, and number of books 
sold for that language and many more factors. 
\\
  It is difficult to decide which programming languages are most widely used. One language may occupy the greater number of 
programmer hours, a different one have more lines of code, a third may utilize the most CPU time, and so on. Some languages are 
very popular for particular kind of applications. For example, COBOL is still strong in the corporate data center, often on large 
mainframes; FORTRAN in engineering applications; C in embedded applications and operating systems; and other languages are 
regularly used to write much different kind of applications.
}
\\
\newpage
\\
\\
\section{\large{methods of measuring programming languages popularity}} \\ \\
\large{
\\ \\A. Counting the number of times the language name is mentioned in web searches, such as is done by Google Trends.
\\ \\B. counting the number of job advertisements that mention the language.
\\ \\C. the number of books sold that teach or describe the language.
\\ \\D. estimates of the number of existing lines of code written in the language – which may underestimate languages not 
often found in public searches. \\ \\E.counts of language references (i.e., to the name of the language) found using a web search engine. \\ \\ F.counting the number of projects in that language on SourceForge, and GitHub
\\ \\G. counting the number of postings in Usenet newsgroups about the language
\\ \\H. comparing the number of commits or changed source lines for open source projects on Open Hub
 }
\begin{center}
\chapter{Reports published on popularity of languages}
\end{center}
\newpageSeveral several indices have been published to decide popularity of programming languages which are given below:
\\ \\The monthly TIOBE Programming Community Index has been published since 2001, and shows the top 10 languages' 
popularity graphically, the top 20 languages with a rating and delta, and the top 50 languages' ratings.The numbers are based 
on searching the Web with certain phrases that include language names and counting the numbers of hits returned.
\\ \\The PYPL PopularitY of Programming Language is an indicator based on Google Trends, reflecting the developers' searches 
for "programming language tutorial", instead of what pages are available.It shows the popularity trends since 2004, 
worldwide or separated for 5 countries.
\\ \\The RedMonk Programming Language Rankings are derived from a correlation of programming traction on GitHub (usage) 
and Stack Overflow (discussion).
\\ \\ The Trendy Skills[16] searches and extracts from popular advertising websites the skills and technologies that employers are 
looking and classifies skills sought in categories, one of which is the Programming Languages category. It allows the user to 
see the trends for one or more skills or categories at specified time ranges. Data is also accessible via a public API, so anyone 
can generate their own statistics.
\\ \\Indeed 2016 survey. Results show that among job advertisements Java is more popular than other languages combined.
\\ \\Stack Overflow's 2016 Developer Survey Results. According to poll JavaScript is used by 55 percantege of developers.
\\ \\ Krihelinator.xyz ranks programming languages based on their github contribution rate according to this formula.
\\ \\IEEE Spectrum's 2016 ranking of top programming languages"
synthesizes 12 metrics from 10 sources to arrive at an 
overall ranking of language popularity".The various metrics were collected from GitHub, Google 
Search and Trends, Twitter, Stack Overflow, Reddit, Hacker News, Career Builder, Dice.com, and IEEE Xplore Digital 
Library. The interactive ranking application allows adjustment of each metric's weight, and also filtering languages by "type" 
(Web, Mobile, Enterprise, Embedded).
\begin{center}
\chapter{TIOBE index} \\
\end{center}
\\
Java and C are in a heavy downward trend since the beginning of 2016. Both languages have lost more than 6 prcentage if compared to 
last year. Other languages are taking advantage of this drop. The TIOBE Programming Community index is an indicator of the 
popularity of programming languages which is updated every month. These ratings are based on the number of skilled engineers 
world-wide, courses and third party vendors. Popular search engines such as Google, Bing, Yahoo!, Wikipedia, Amazon, 
YouTube and Baidu are used to calculate these ratings. The index can be used to check whether your programming skills are still 
up to date or to make a strategic decision about what programming language should be adopted when starting to build a new 
software system.
\\
\\
\begin{center}
\\
To see the bigger picture, please find below the positions of the top 10 programming languages of many years back. These are 
average positions for a period of 12 months.
\end{center}
\begin{center}
\begin{tabular}{||c|c|c|c|c|c|c|c||}
   Programming Language & 2017& 2012& 2007& 2002& 1997& 1992& 1987\\
Java& 1& 1& 1& 1 &14& -& -\\
C &2 &2& 2& 2& 1& 1& 1\\
C++& 3& 3& 3& 3& 2& 2& 4\\
C# &4 &4 &7 &14 &- &- &-\\
Python& 5 &7 &6 &9 &27 &- &-\\
PHP &6 &5 &4 &5 &- &- &-\\
JavaScript& 7& 9& 8& 7& 20& -& -
\\Visual Basic .NET &8 &21& -& - &- &-& -
\\Perl& 9& 8& 5& 4& 4& 11& -
\\Assembly language &10 &-& - &- &- &- &-
\\COBOL &25 &31 &17& 6& 3 &13& 8
\\Lisp &31& 12 &14& 10& 9 &9 &2
\\Prolog &33 &37 &26 &13 &18 &14 &3
\\Pascal& 102& 13& 19& 29& 8 &3 &5\\
\end{tabular}
\end{center}
\\
\\
\\
\\
\begin{center}
    \\
    \\
    \\
\end{center}
\begin{center}
    \\
    \\
    \\
\end{center}

\begin{center}
\\
\begin{tabular}{||c|c||}
2003&C++\\ 2004&PHP \\2005&Java \\ 2006&Ruby\\  2007&Python\\ 2008&C\\ 2009&Go\\ 2010&Python \\2011&Objective-C\\ 2012&Objective-C\\ 2013&Transact-SQL\\ 2014&JavaScript \\2015 &Java \\ 2016&Go
\\      
\end{tabular}
\begin{center}
    \\
    \\
    \\
\end{center}
The hall of fame listing all "Programming Language of the Year" award winners is shown below. The award is given to the 
programming language that has the highest rise in ratings in a year. 
\end{center}


\begin{center}
\chapter{the redmonk programming language rankings} 
\end{center}
The data was dutifully collected and analyzed at RedMonk. Periodically, the performance of programming languages compared
relative to one another on GitHub and Stack Overflow. The idea is not to offer a statistically valid representation of current usage, 
but rather to correlate language discussion (Stack Overflow) and usage (GitHub) in an effort to extract insights into potential 
future adoption trends. With the exception of GitHub’s decision to no longer provide language rankings on its Explore page – they 
are now calculated from the GitHub archive – the rankings are performed in the same manner, meaning that we can compare
rankings year to year, with confidence.
Historically, the correlation between how a language ranks on GitHub versus its ranking on Stack Overflow has been strong, but 
this had been weakening in recent years. From its highs of .78, the correlation was down to .73 during our last run – the lowest 
recorded. For this run, however, the correlation between the properties is once again robust. For this quarter’s ranking, the
correlation between the properties was .77, just shy of its all time mark. Given the recent variation, however, it will be interesting 
to observe whether or not this number continues to bounce.
Before we continue, please keep in mind the usual caveats.
\\ \\ To be included in this analysis, a language must be observable within both GitHub and Stack Overflow.
\\ \\ The examination of the correlation between two populations is to be predictive of future use, hence their value.
\\ \\ GitHub and Stack Overflow are surveyed here first because of their size and second because of their public exposure of 
the data necessary for the analysis. 
\\ \\The numerical ranking is substantially less relevant than the language’s tier or grouping. In many cases, one spot on the 
list is not distinguishable from the next. The separation between language tiers on the plot, however, is generally 
representative of substantial differences in relative popularity.
\\ \\ GitHub language rankings are based on raw lines of code, which means that repositories written in a given language that 
include a greater amount of code in a second language (e.g. JavaScript) will be read as the latter rather than the former.
\\ \\ In addition, less data is available to rank languages further down the rankings so the actual placement of languages 
becomes less reliable down the list.\\
\begin{center}
    \includegraphics{image.png}
\end{center}
Figure: 4.1\\
\begin{center}
    \includegraphics[scale=0.85]{dd.jpeg}
\end{center}
Figure:4.2\\
We offer the following numerical rankings. As will be observed, this run produced several ties which are reflected below (they 
are listed out here alphabetically rather than consolidated.
\\1. JavaScript
\\2. Java
\\3. PHP
\\4. Python
\\5. C#
\\6. C++
\\7. Ruby
\\8. CSS
\\9. C
\\10. Objective-C
\\11. Shell
\\12. Perl
\\13. R
\\14. Scala
\\15. Go
\\16. Haskell
\\17. Swift
\\18. Matlab
\\19. Clojure
\\20. Groovy
\\21. Visual Basic
JavaScript’s continued strength is impressive, as is Java’s steady, robust performance. The long time presence of these two 
languages in particular atop our rankings is no coincidence; instead it reflects an increasing willingness to employ a best-tool-forthe-job approach, even within the most conservative of enterprises. In many cases, Java and JavaScript are leveraged side-by-side 
in the same application, depending on its particular needs. Just as JavaScript and Java’s positions have remained unchanged, the 
rest of the Top 10 has remained similarly static. This has become the expectation rather than a surprise. As with businesses, the 
larger a language becomes, the more difficult it is to outperform from a growth perspective. This suggests that what changes we’ll 
see in the Top 10 will be slow and longer term, that fragmentation has begun to slow. The two most obvious candidates for a Top 
10 ranking at this point appear to be Go and Swift, but they have their work cut out for them before they get there.

    \chapter{Glassdor Programming Language Ranking}
    \\
    \\
    Glassdoor recently published a report on the top 25 lucrative, in-demand jobs. More than half of the jobs listed are in tech and 
require programming skills. For a fast-growing and lucrative career, everyone wants to make learning to best programming 
language. We compiled data from Indeed.com (database including current computer programmer jobs). While this isn’t an 
extensive list, it does provide insight into the most in-demand programming languages sought after by employers.\\
\begin{center}
    \includegraphics[scale=0.75]{kk.png}\\
    figure:5.1\\
\end{center}\\
\begin{center}
    \includegraphics[scale=0.80]{Screenshot_20230115_115641.png}\\
    figure:5.2\\
\end{center}
\begin{center}
    \includegraphics[scale=0.75]{R.png}\\
    figure:5.3\\
\end{center}\\
\begin{center}
\chapter{conclusion}
In conclusion, Java Programming Language is maintaining its first position where as C and C++ is at positions 2nd and 3rd
respectively since last 15 years. Java Script and SQL cannot be considered as programming languages but these are occupying top 
positions in various rankings depending upon number of jobs available in the market and number of searches at Google Trends 
and various search Engines. Python, PHP and Ruby are also gaining grounds in this dynamic world of programming languages. 
\end{center}
\chapter{Bibliography} 

\\
Tool used :
\\www.overleaf.com
\\
\\
[1] "SSL/Computer Weekly IT salary survey: finance boom drives IT job growth". ComputerWeekly.com. September 2007. Retrieved 14
June 2013.
\\[2] "Jobs Tractor language trends, based on jobs advertised on Twitter". JobsTractor. Retrieved 14 June 2013.
\\[3] O'Reilly, Tim. "Programming Language Trends". O'Reilly Radar. Retrieved 14 June 2013.
\\[4] "State of the Computer Book Market 2008, part 4 — The Languages - O'Reilly Radar". Radar.oreilly.com. 2009-02-25.
Retrieved 2017-03-14.
\\[5] Bieman, J.M.; Murdock, V., Finding code on the World Wide Web: a preliminary investigation, Proceedings First IEEE International
Workshop on Source Code Analysis and Manipulation, 2001
\\[6] "Tiobe Index Definition". TIOBE Software. Retrieved 10 April 2012.
1993 1998 2003 2008
ADA C C++
FORTRAN JAVA PASCAL
SMALLTALK
\\[7] "Programming Language Usage Graph". Wismuth.com. 2010-10-31. Retrieved 2017-03-14.
\\[8] "Trends for the Future". Catb.org. Retrieved 2017-03-14.
\\[9] "Language Trends on GitHub · GitHub". github.com. 2015-08-19. Retrieved 2017-03-14.
\\[10] "Programming language popularity". Complang.tuwien.ac.at. Retrieved 2017-03-14.
\\[11] "Compare Languages". Open Hub. Retrieved 2017-01-20.
\\[12] "TIOBE Programming Community Index". TIOBE Software BV. Retrieved 14 June 2013.
\\[13] "PYPL PopularitY of Programming Language index". Pypl.github.io. 2013-11-22. Retrieved 2017-03-14.
\\[14] "PYPL PopularitY of Programming Language index". Pypl.github.io. 2013-11-22. Retrieved 2017-03-14.
\\[15] O'Grady, Stephen (2016-02-19). "The RedMonk Programming Language Rankings: January 2016". Redmonk.com. Retrieved 2017-
03-14.\\
\begin{center}
    *****
\end{center}
\end{document}